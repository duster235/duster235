\chapter{Tagebuch}
%\label{ch:intro}

\section{Woche 1: 4.4. bis 8.4.}
Es wurden diverse Programme installiert sowie Rechte und Lizenzen angefordert. Aufgabe: For licensing reasons we must know the number of unique people that trigger out jenkins toolchain each month. That should be automated and must generate a report in a format to be defined with Jürgen Sammet by Dathe, Patrick. Es wurden erste Versuche unnternommen mit der Github Rest API zu arbeiten und Daten zu erhalten. Programmiersprache Python. Versionierngssoftware git mit git extensions.

\section{Woche 2: 11.4 bis 15.4.}
Weiter an Projekt gearbeitet. 
\section{Woche 3: 18.4 bis 22.4.}

\section{Woche 4: 25.4 bis 29.4.}
\subsection{Montag}
Das Jenkinstool wurde fertiggestellt. Es gibt nun 2 Jobs die jeweils 1mal im Monat gebuildet werden. Es wurde angepasst, dass user ohne eine ui Nummer mit aufgenommen werden. Außerdem wurde eine Blacklist implementiert, die alte usernamen oder Duplikate entfernt. 

Es wurde das RFID Handheld Lesegerät in Kombination mit dem Smartrac Dogbone Temperatursensor in betrieb genommen. Einlesen in die Nutzung der NordicID NUR API.

\subsection{Dienstag}
\begin{itemize}
	\item Lesen alter Bachelorarbeiten und einlesen in Thermos confluence
\end{itemize}

\subsection{Mittwoch}
\begin{itemize}
	\item started working on tsw-23 simulink model advisor automation with jenkins pipeline
	\item documentation of reading RFID sensors on confluence \ac{CES}
\end{itemize}

\section{Woche 5: 2.5 bis 6.5.}
\subsection{Donnerstag}
\begin{itemize}
	\item 
\end{itemize}